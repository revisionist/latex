
% Define commands for metadata variables
\def\titlemain{}
\def\titlesub{}
\def\author{}
\def\authorsurname{}
\def\authorforename{}
\def\faculty{}
\def\institution{}
\def\institutionlogo{}
\def\submitdate{}
\def\degree{}
\def\supervisor{}
\def\draftmode{false} % Default value for draft mode

% Set the date for the document
\date{\today}

% Set the detault resource root.  This is the relative path for including things like images,
% which we can optionally include in \includegraphics commands etc.
\newcommand{\resourceroot}{.}

% Define some commonly used commands for a thesis
\newcommand{\ipc}{{\sf ipc}}

% Define mathematical symbols
\newcommand{\Prob}{\bbbp}
\newcommand{\Real}{\bbbr}
\newcommand{\real}{\Real}
\newcommand{\Int}{\bbbz}
\newcommand{\Nat}{\bbbn}

% More mathematical symbols for various number sets
\newcommand{\NN}{{\sf I\kern-0.14emN}}	% Natural numbers
\newcommand{\ZZ}{{\sf Z\kern-0.45emZ}}	% Integers
\newcommand{\QQQ}{{\sf C\kern-0.48emQ}}	% Rational numbers
\newcommand{\RR}{{\sf I\kern-0.14emR}}	% Real numbers

% Define commands for mathematical notation and variables
\newcommand{\KK}{{\cal K}}
\newcommand{\OO}{{\cal O}}
\newcommand{\AAA}{{\bf A}}
\newcommand{\HH}{{\bf H}}
\newcommand{\II}{{\bf I}}
\newcommand{\LL}{{\bf L}}
\newcommand{\PP}{{\bf P}}
\newcommand{\PPprime}{{\bf P'}}
\newcommand{\QQ}{{\bf Q}}
\newcommand{\UU}{{\bf U}}
\newcommand{\UUprime}{{\bf U'}}
\newcommand{\zzero}{{\bf 0}}
\newcommand{\ppi}{\mbox{\boldmath $\pi$}}
\newcommand{\aalph}{\mbox{\boldmath $\alpha$}}
\newcommand{\bb}{{\bf b}}
\newcommand{\ee}{{\bf e}}
\newcommand{\mmu}{\mbox{\boldmath $\mu$}}
\newcommand{\vv}{{\bf v}}
\newcommand{\xx}{{\bf x}}
\newcommand{\yy}{{\bf y}}
\newcommand{\zz}{{\bf z}}
\newcommand{\oomeg}{\mbox{\boldmath $\omega$}}
\newcommand{\res}{{\bf res}}
\newcommand{\cchi}{{\mbox{\raisebox{.4ex}{$\chi$}}}}
%\newcommand{\cchi}{{\cal X}}
%\newcommand{\cchi}{\mbox{\Large $\chi$}}

% Logical operators and symbols
\newcommand{\imply}{\Rightarrow}
\newcommand{\bimply}{\Leftrightarrow}
\newcommand{\union}{\cup}
\newcommand{\intersect}{\cap}
\newcommand{\boolor}{\vee}
\newcommand{\booland}{\wedge}
\newcommand{\boolimply}{\imply}
\newcommand{\boolbimply}{\bimply}
\newcommand{\boolnot}{\neg}
\newcommand{\boolsat}{\!\models}
\newcommand{\boolnsat}{\!\not\models}

% Operator and set notation
\newcommand{\op}[1]{\mathrm{#1}}
\newcommand{\s}[1]{\ensuremath{\mathcal #1}}

% Properly styled differentiation and integration operators
\newcommand{\diff}[1]{\mathrm{\frac{d}{d\mathit{#1}}}}
\newcommand{\diffII}[1]{\mathrm{\frac{d^2}{d\mathit{#1}^2}}}
\newcommand{\intg}[4]{\int_{#3}^{#4} #1 \, \mathrm{d}#2}
\newcommand{\intgd}[4]{\int\!\!\!\!\int_{#4} #1 \, \mathrm{d}#2 \, \mathrm{d}#3}

% Large () brackets on different lines of an eqnarray environment
\newcommand{\Leftbrace}[1]{\left(\raisebox{0mm}[#1][#1]{}\right.}
\newcommand{\Rightbrace}[1]{\left.\raisebox{0mm}[#1][#1]{}\right)}

% Funky symobols for footnotes
\newcommand{\symbolfootnote}{\renewcommand{\thefootnote}{\fnsymbol{footnote}}}
% now add \symbolfootnote to the beginning of the document...

% Line spacing commands
\newcommand{\normallinespacing}{\renewcommand{\baselinestretch}{1.5} \normalsize}
\newcommand{\mediumlinespacing}{\renewcommand{\baselinestretch}{1.2} \normalsize}
\newcommand{\narrowlinespacing}{\renewcommand{\baselinestretch}{1.0} \normalsize}


% Miscellaneous formatting commands
\newcommand{\bump}{\noalign{\vspace*{\doublerulesep}}}
\newcommand{\cell}{\multicolumn{1}{}{}}
\newcommand{\spann}{\mbox{span}}
\newcommand{\diagg}{\mbox{diag}}
\newcommand{\modd}{\mbox{mod}}
\newcommand{\minn}{\mbox{min}}
\newcommand{\andd}{\mbox{and}}
\newcommand{\forr}{\mbox{for}}
\newcommand{\EE}{\mbox{E}}

% Definition and synchronisation commands
\newcommand{\deff}{\stackrel{\mathrm{def}}{=}}
\newcommand{\syncc}{~\stackrel{\textstyle \rhd\kern-0.57em\lhd}{\scriptstyle L}~}
\def\coop{\mbox{\large $\rhd\!\!\!\lhd$}}
\newcommand{\sync}[1]{\raisebox{-1.0ex}{$\;\stackrel{\coop}{\scriptscriptstyle #1}\,$}}

% Theorem and definition environments

\newtheoremstyle{indentedstyle}  % The name of the custom style
{10pt}      % Space above
{10pt}      % Space below
{\normalfont}  % Body font (normal text)
{2em}       % Indent amount for the whole hypothesis
{\bfseries} % Theorem head font (bold for "Hypothesis")
{:}         % Punctuation after theorem header
{ }         % Space after theorem header
{\thmname{#1}\ \thmnumber{#2}}  % Theorem heading ("Hypothesis X.")

\theoremstyle{indentedstyle}

\makeatletter
\@ifclassloaded{book}{

	\newtheorem{definition}{Definition}[chapter]
	\newtheorem{theorem}{Theorem}[chapter]
	%\newtheorem{theorem}{Theorem}[section]
	\newtheorem{corollary}{Corollary}[theorem]
	\newtheorem{lemma}[theorem]{Lemma}
	%\newtheorem{hypothesis}{Hypothesis}[chapter]
	\newtheorem{hypothesis}{Hypothesis}

	\newtheorem{researchquestion}{Research Question}
	\renewcommand{\theresearchquestion}{RQ\arabic{researchquestion}}
	
	\newtheorem{subquestion}{Sub-Question}[researchquestion]
	\renewcommand{\thesubquestion}{\theresearchquestion\alph{subquestion}}
	
	\renewcommand{\thehypothesis}{H\arabic{hypothesis}}

}{}
\makeatother








% References to figures
\newcommand{\Figref}[1]{Figure~\ref{#1}}
\newcommand{\fig}[3]{
	\begin{figure}[!ht]
		\begin{center}
			\scalebox{#3}{\includegraphics{figs/#1.ps}}
			\vspace{-0.1in}
			\caption[ ]{\label{#1} #2}
		\end{center}
	\end{figure}
}

% Two figures side by side
\newcommand{\figtwo}[8]{
	\begin{figure}
		\parbox[b]{#4 \textwidth}{
			\begin{center}
				\scalebox{#3}{\includegraphics{figs/#1.ps}}
				\vspace{-0.1in}
				\caption{\label{#1}#2}
			\end{center}
		}
		\hfill
		\parbox[b]{#8 \textwidth}{
			\begin{center}
				\scalebox{#7}{\includegraphics{figs/#5.ps}}
				\vspace{-0.1in}
				\caption{\label{#5}#6}
			\end{center}
		}
	\end{figure}
}

% Correct bad hyphenation
\hyphenation{op-tical net-works semi-conduc-tor}

% Invisible section to add to TOC without showing in document
\newcommand\invisiblesection[1]{%
	\refstepcounter{section}%
	\addcontentsline{toc}{section}{\protect\numberline{\thesection}#1}%
	\sectionmark{#1}}

% Handle latin abbreviations.  Slightly modified from:
% https://stackoverflow.com/questions/3282319/correct-way-to-define-macros-etc-ie-in-latex/3285603#3285603
\ExplSyntaxOn % Uses expl3
\newcommand\latinabbrev[1]{
	\peek_meaning:NTF . {% Same as \@ifnextchar
		#1\@}%
	{ \peek_catcode:NTF a {% Check whether next char has same catcode as \'a, i.e., is a letter
			#1.\@\xspace}%
		{#1.\@}}}
\ExplSyntaxOff

% Define latin abbreviations
\def\eg{\latinabbrev{\textit{e}.\textit{g}}}
\def\etal{\latinabbrev{\textit{et al}}}
\def\etc{\latinabbrev{\textit{etc}}}
\def\ie{\latinabbrev{\textit{i}.\textit{e}}}
\def\vs{\latinabbrev{\textit{vs}}}

% Insert a TODO marker, either with a colon after it if text follows
% or without one if a period follows:
%  \todo Some text			-> TODO: Some text
%  Preceding text \todo.	-> Preceding text TODO.
\ExplSyntaxOn % Uses expl3
\newcommand\todocmd[1]{
	\peek_catcode:NTF . {#1} {#1:\xspace}
}
\ExplSyntaxOff
\def\todo{\todocmd{\textbf{TODO}}}

% Define attribution commands
\newcommand{\attributionfootnote}[1]{\footnotemark\footnotetext{Attribution: #1}}
\newcommand{\attributioncaption}[1]{\\ \footnotesize{Attribution: #1}}

% Default attribution command (set to footnote or caption as needed)
\newcommand{\attribution}{\attributionfootnote} % or \attributioncaption

% Chapter abstract box
\newtcolorbox{chapterabstract}[1][]{
	colback=white,
	colframe=black,
	fonttitle=\bfseries,
	title=#1,
	width=\textwidth,
	left=1cm,
	right=1cm,
	coltitle=black,
	fontupper=\itshape,
	colbacktitle=white,
	coltitle=black,
	colupper=black,
	enhanced,
	boxrule=0.5pt,
	colframe=gray,
	sharp corners,
}

% Custom spacing for lists
\setlist[itemize]{topsep=0mm, partopsep=0mm, parsep=0mm, itemsep=2mm}
\setlist[enumerate]{topsep=0mm, partopsep=0mm, parsep=0mm, itemsep=2mm}

% Define ellipsis spacing
\renewcommand{\ellipsisgap}{0.1em}

% Flowcharts etc.
\usetikzlibrary{shapes.geometric, arrows, calc}


% Create hyperlinks with refs
\newcommand{\hyperrefWithRefBrackets}[2]{%
	[\ref{#1}]\enspace\hyperref[#1]{#2}%
}
\newcommand{\hyperrefWithRef}[2]{%
	\ref{#1}\enspace--\enspace\hyperref[#1]{#2}%
}

% Create URLs with perma.cc links
\newcommand{\urlWithPerma}[2]{%
	\url{#1}~[\url{https://perma.cc/#2}]%
}
\newcommand{\footnoteUrlWithPerma}[2]{%
	\footnote{\urlWithPerma{#1}{#2}}%
}

% Create DOI hyperrefs
\newcommand{\hyperrefDoi}[2]{%
	\href{https://doi.org/#1}{\textcolor{#2}{\texttt{doi:#1}}}%
}
\newcommand{\hyperrefDoiWithPathXX}[2]{%
	~{\footnotesize \textsf{DOI: }}\hyperrefDoi{#1}{hyperdarkerblue}~{\footnotesize \textsf{[path: \texttt{\detokenize{#2}}]}}%
}
\newcommand{\hyperrefDoiWithPath}[2]{%
	\hyperrefDoi{#1}{hyperdarkerblue}~{\footnotesize \textsf{[path: \texttt{\detokenize{#2}}]}}%
}


% Used for some logical operations
\newcommand{\true}{true}
\newcommand{\false}{false}

\definecolor{orcidlogocol}{HTML}{A6CE39}
\definecolor{headergrey}{HTML}{999999}
\definecolor{darkred}{HTML}{8B0000}

% Set global format for datetime2
\DTMsetdatestyle{iso} % Use ISO style for date
\DTMsetup{
	datesep=.,
	timesep=:,
	showseconds=false,
	showzone=false
}

% Set colours for hyperlinks
% https://tex.stackexchange.com/questions/823/remove-ugly-borders-around-clickable-cross-references-and-hyperlinks
%\hypersetup{
	%	colorlinks = true,
	%	linkcolor={red!40!black},
	%	citecolor={blue!40!black},
	%	urlcolor={blue!50!black}
	%}

\definecolor{hyperblue}{rgb}{0.1, 0.1, 0.5}
\definecolor{hyperdarkerblue}{rgb}{0.0, 0.1, 0.4}

\hypersetup{
	colorlinks = true,
	linkcolor = hyperdarkerblue,
	citecolor = hyperblue,
	urlcolor  = hyperblue,
	filecolor = hyperblue
}


% Listings styles for Python and Java

\definecolor{eclipseStrings}{RGB}{42,0.0,255}
\definecolor{eclipseKeywords}{RGB}{127,0,85}
\colorlet{numb}{magenta!60!black}

\lstdefinestyle{pythonStyle}{
	language=Python,
	basicstyle=\ttfamily\scriptsize,
	keywordstyle=\color{blue}\bfseries,
	stringstyle=\color{teal},
	commentstyle=\color{gray},
	tabsize=3,
	breakindent=1.5em,
	showstringspaces=false,
	breaklines=true,
	frame=single,
	captionpos=b
}

\lstdefinestyle{javaStyle}{
	language=Java,
	basicstyle=\ttfamily\scriptsize,
	keywordstyle=\color{purple}\bfseries,
	stringstyle=\color{orange},
	commentstyle=\color{gray},
	tabsize=3,
	breakindent=1.5em,
	showstringspaces=false,
	breaklines=true,
	frame=single,
	captionpos=b
}

\lstdefinestyle{jsonStyle}{
	basicstyle=\ttfamily\scriptsize,
	keywordstyle=\color{blue}\bfseries,
	commentstyle=\color{eclipseStrings}, % style of comment
	stringstyle=\color{eclipseKeywords}, % style of strings
	numberstyle=\scriptsize,
	showstringspaces=false,
	tabsize=3,
	breakindent=1.5em,
	stepnumber=1,
	numbersep=8pt,
	breaklines=true,
	frame=single,
	captionpos=b,
	string=[s]{"}{"},
	comment=[l]{:\ "},
	morecomment=[l]{:"},
	literate=
	*{0}{{{\color{numb}0}}}{1}
	{1}{{{\color{numb}1}}}{1}
	{2}{{{\color{numb}2}}}{1}
	{3}{{{\color{numb}3}}}{1}
	{4}{{{\color{numb}4}}}{1}
	{5}{{{\color{numb}5}}}{1}
	{6}{{{\color{numb}6}}}{1}
	{7}{{{\color{numb}7}}}{1}
	{8}{{{\color{numb}8}}}{1}
	{9}{{{\color{numb}9}}}{1}
}



\noindent
\phantomsection
\begin{minipage}[t]{0.15\textwidth}
	\centering
	\textbf{Type 1} \par
	\textit{Category A} \par
	\label{example-1:1.a}
\end{minipage}
\hspace{0.5cm}
\begin{minipage}[t]{0.8\textwidth}
	\textbf{This is the title of the item} \par
	This is some text.  In this case we have defined a couple of footnotes\footnote{Like this one} to show how they are shown when we have minipages formatted like this.  We can also use this type of footnote\footnoteUrlWithPerma{https://xkcd.com/327/}{EP49-TCM6}.
\end{minipage}
\vspace{0.5cm}

\noindent
\phantomsection
\begin{minipage}[t]{0.15\textwidth}
	\centering
	\textbf{Type 1} \par
	\textit{Category B} \par
	\label{example-1:1.b}
\end{minipage}
\hspace{0.5cm}
\begin{minipage}[t]{0.8\textwidth}
	\textbf{This is the title of the item} \par
	\lipsum[4]
\end{minipage}
\vspace{0.5cm}

\noindent
\phantomsection
\begin{minipage}[t]{0.15\textwidth}
	\centering
	\textbf{Type 2} \par
	\textit{Category C} \par
	\label{example-1:2.c}
\end{minipage}
\hspace{0.5cm}
\begin{minipage}[t]{0.8\textwidth}
	\textbf{This is also a title} \par
	\lipsum[5]
\end{minipage}
\vspace{0.5cm}
